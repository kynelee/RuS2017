\documentclass{article}
\usepackage[margin=1.0in]{geometry}

\begin{document}
\noindent
Matthew Lee\\
Principles of Programming Languages Homework 1\\
Section 02\\
\section{}
  \subsection{$f(x) = x + 1, x > 0$}
    \paragraph{Rules}
      \begin{enumerate}
        \item $\$ \rightarrow |$
        \item $|\# \rightarrow |$
      \end{enumerate}
    \paragraph{Sequence}
      \begin{enumerate}
        \item $\$|||\# \rightarrow ||||\#$ (Rule 1)
        \item $||||\# \rightarrow ||||$ (Rule 2)
      \end{enumerate}
  \subsection{$f(x) = 2x, x > 0$}
    \paragraph{Rules}
      \begin{enumerate}
        \item $| \rightarrow ||$
        \item $\$| \rightarrow |$
        \item $|\# \rightarrow |$
      \end{enumerate}
    \paragraph{Sequence}
      \begin{enumerate}
        \item $\$|||\# \rightarrow \$||||||\#$ (Rule 1)
        \item $\$||||||\# \rightarrow ||||||\#$ (Rule 2)
        \item $||||||\# \rightarrow ||||||$ (Rule 3)
      \end{enumerate}
  \subsection{$ f(x, y) = x + y, x,y > 0 $}
    \paragraph{Rules}
      \begin{enumerate}
        \item $ \& \rightarrow NULL $
        \item $ \$| \rightarrow | $
        \item $ |\# \rightarrow | $
      \end{enumerate}
    \paragraph{Sequence}
      \begin{enumerate}
        \item $ \$||\&|||\# \rightarrow \$||||| $(Rule 1)
        \item $\$|||||\# \rightarrow |||||\# $(Rule 2)
        \item $|||||\# \rightarrow ||||| $ (Rule 3)
      \end{enumerate}
\section{}
  \subsection{Rules}
      \begin{enumerate}
        \item $ (0 + 0)  \rightarrow 0 $
        \item $ (0 + 1) \rightarrow 1 $
        \item $ (1 + 0) \rightarrow 1 $
        \item $ (0 + 2) \rightarrow 2 $
        \item $ (2 + 0) \rightarrow 2 $
        \item $ (1 + 1) \rightarrow 2 $
        \item $ (1 + 2) \rightarrow 0 $
        \item $ (2 + 1) \rightarrow 0 $
        \item $ (2 + 2) \rightarrow 1 $
      \end{enumerate}
  \subsection{Sequence Example}
      \paragraph{Sequence 1: $(0+(1+2))$}
        \begin{enumerate}
          \item $ ((1+2) + 0) \rightarrow (0 + 0) $ (Rule 7)
          \item $ (0 + 0) \rightarrow 0 $ (Rule 1) 
        \end{enumerate}
      \paragraph{Sequence 2: $(1+(2+2))$}
        \begin{enumerate}
          \item $ (1 + (2 + 2)) \rightarrow (1 + 1) $ (Rule 9)
          \item $ (1 + 1) \rightarrow 2 $ (Rule 6)
        \end{enumerate}
  \subsection{}
    No, in $ ((1 + 1) + (2+2)) $ 2 rules can be applied

\section{}
    \subsection{}
      Yes: $ r^+ = r\&r^* $
    \subsection{}
      No: At least one of those operators is necessary to express an
      infinite quantity
\section{}
  $ R = (digit)^+.(digit)^+(\epsilon(digit)^+ | \epsilon) $

\section{}
  \subsection{}
    All strings/combinations over the alphabet $ {a,b} $ including the empty string
  \subsection{}
    All strings over the alphabet $ {0, 1} $ which begin with 1 and end in
    001 or 011
\section{}
  \subsection{}
    $ R = (b^*c | a | c | d)^*b^* $
  \subsection{}
    $ R = a^*(b | \epsilon)a^*(c | \epsilon)a^*(c | \epsilon)a^*(c | \epsilon)a^*(c | \epsilon)a^* %
    \\ | a^*(c | \epsilon)a^*(b | \epsilon)a^*(c | \epsilon)a^*(c | \epsilon)a^*(c | \epsilon)a^* %
    \\ | a^*(c | \epsilon)a^*(c | \epsilon)a^*(b | \epsilon)a^*(c | \epsilon)a^*(c | \epsilon)a^* %
    \\ | a^*(c | \epsilon)a^*(c | \epsilon)a^*(c | \epsilon)a^*(b | \epsilon)a^*(c | \epsilon)a^* %
    \\ | a^*(c | \epsilon)a^*(c | \epsilon)a^*(c | \epsilon)a^*(c | \epsilon)a^*(b | \epsilon)a^* $
\end{document}
